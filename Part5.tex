
\section{Discrete Kalman Filter} \label{sec:part5}
In this part a discrete Kalman filter is implemented to estimate the bias b and the heading $\psi$. The high-frequented wave induced motion $\psi_m$ is also estimated, but this must be removed from the control loop to avoid wear and tear on the actuator system.  

We continue to be careful not to make $\psi$ bigger than $\pm 35 \circ$ and small deviations in compass value. \todo{husk å nevne i d og e}

\subsection{Exact discretization}
To use a discrete Kalman filter, a discretized system is needed. The matrices A, B and E found in \ref{sec:part4-1} was put into MATLAB and the function a2d() was used to get an exact discretization of the system. 


\subsection{Estimate of measurement noise variance}
To find the estimate of the variance of the measurement noise we used the Matlab function \texttt{var}.
\newline

\label{eq:noise_vaiance} %label til varaince of the measurement noise i 5.5.b)

MATLAB KODE

\subsection{Implementation of discrete Kalman filter}
Now we let 
\begin{equation}
    \boldsymbol{w} = [w_w \ w_b]^T \ , \ E\{\boldsymbol{w w^T}\} = \boldsymbol{Q} = \begin{bmatrix}
        30 & 0 \\
        0 & 10^{-6} 
    \end{bmatrix}
\end{equation}

\begin{equation}
    \boldsymbol{P_0^-} = \begin{bmatrix}
        1 & 0 & 0 & 0 & 0 \\
        0 & 0.013 & 0 & 0 & 0 \\
        0 & 0 & \pi^2 & 0 & 0 \\
        0 & 0 & 0 & 1 & 0 \\
        0 & 0 & 0 & 0 & 2.5 \cdot 10^{-3} 
    \end{bmatrix} \ , \ \boldsymbol{x_0^-} = \begin{bmatrix}
        0 \\ 0 \\ 0 \\ 0 \\ 0 \\
    \end{bmatrix}
\end{equation}
where \textbf{w} is the process noise, \textbf{Q} is the process noise covariance and $\boldsymbol{\hat{x}_0^-}$ is the initial a priori state estimate. Sice the process is sampled, $E(v^2) = R$ equals the measurement noise variance found in \cref{eq:noise_vaiance} divided by the sample interval. 
\newline

Implemented the dicrete Kalman filter using ... En av to metoder. \newline

Fra OPPGAVETEKST: Traditionally, s-functions are better suited for dynamic systems since they facilitate initialization, termination and updates at a given frequency. The Matlab function block is designed for static systems, so it does not have the same overhead as s-functions. This makes Matlab function seem easier, but it also require some additional work to set up the initialization of the Kalman filter. Hint: Let the compass measurement and the rudder input be input to the Kalman filter function and the output be the a posteriori estimate of $\psi$ and b. Also put a zero order hold on the compass measurement and the rudder command, since the process and controller are continuous. Use the same sampling frequency as above. Use the update expression (4.2.11) in Brown\&Hwang 


\subsection{Simulation with current disturbance with feed forward}
Using $\psi_r = 30$ and making a feed forward from the estimated bias such that the bias is cancelled makes ... Simulating the system with a current disturbance, makes the autopilot have a BETTER ?? performance than the equivalent simulation in 5.3.c) LABEL? ELABORATE
\newline

INCLUDE PLOTS OF BOTH MEASURED AND FILTERED COMPASS COURSE, RUDDER INPUT AND ESTIMATED BIAS. AND PLOTS OF ACTUAL WAVE INFLUENCE AND ESTIMATED WAVE INFLUENCE.

\subsection{Simulation with wave and current disturbance and wave filtering}
Here we use the wave filtered $\psi$ instead of the measured heading in the autopilot. Simulating the system with both wave and current disturbance, gives .?.?... The autopilot performs BETTER?? than the equivalent simulation in problem 5.3.d). ELABORATE
\newline

INCLUDE PLOTS OF BOTH MEASURED AND FILTERED COMPASS COURSE, RUDDER INPUT AND ESTIMATED BIAS. AND PLOTS OF ACTUAL WAVE INFLUENCE AND ESTIMATED WAVE INFLUENCE.